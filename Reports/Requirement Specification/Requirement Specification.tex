\documentclass[coverpage]{../custom}
\usepackage[dvipsnames]{xcolor}
\usepackage{ctable}
\usepackage{tabularx}
\usepackage{makecell}
\renewcommand{\cellalign}{lc}
\newcolumntype{L}[1]{>{\hsize=#1\hsize\raggedright\arraybackslash}X}
\newcolumntype{R}[1]{>{\hsize=#1\hsize\raggedleft\arraybackslash}X}

\newcommand{\FunctionalReq}[6]{
\begin{tabularx}{\textwidth}{|L{0.55}|R{1.45}|}
	\hline
	\textbf{ID, type, and title} & \textbf{\makecell*{#1}} \\\specialrule{.12em}{.05em}{.05em}
	\textbf{Description} & \makecell*{#2} \\\hline
	\textbf{\makecell*{{\color{red}Mu\color{orange}Sh\color{Green}Co} - Priority}} & \makecell*{#3} \\\hline
	\textbf{Dependencies} & \makecell*{#4} \\\hline
	\textbf{Expected Results} & \makecell*{#5} \\\hline
	\textbf{Exception handling} & \makecell*{#6} \\\hline
\end{tabularx}
}

\Title{Requirement Specification}

\begin{document}
\maketitle

\tableofcontents\newpage

\section{Introduction}
\subsection{Overview and Justification}

\subsection{Project Scope}
\subsection{System Description}
\section{Solution Requirements}
\subsection{Functional Requirements}
\FunctionalReq{ID, Type, Title}{Description\\Over multiple lines}{Priority}{No dependencies}{Results}{let it all crash}

\subsection{Non-functional Requirements}
\subsection{Risks and Issues}
\section{Project Development}
\subsection{Development Approach}
\subsection{Project Schedule}

\end{document}
